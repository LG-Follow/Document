\documentclass[conference]{IEEEtran}
\usepackage{titlesec}
\usepackage\{array\} 
\usepackage{float}
\usepackage{graphicx}
\usepackage{caption}
\usepackage{article}

\ifCLASSINFOpdf
\hyphenation{op-tical net-works semi-conduc-tor}


\begin{document}

\title{LG Follow}


\author{\IEEEauthorblockN{Mingyu Jung}
\IEEEauthorblockA{\textit{Dept. of Information Systems} \\
\textit{Hanyang Univ.}\\
Seoul, Republic of Korea\\
alsrb595@hanyang.ac.kr
}
\and
\IEEEauthorblockN{Taegeon Park}
\IEEEauthorblockA{\textit{Dept. of Information Systems} \\
\textit{Hanyang Univ.}\\
Seoul, Republic of Korea \\
qkrxorjs@hanyang.ac.kr
}
\and
\IEEEauthorblockN{Gyudong Kim}
\IEEEauthorblockA{\textit{Dept. of Information Systems} \\
\textit{Hanyang Univ.}\\
Seoul, Republic of Korea \\
gyudong1594@hanyang.ac.kr
}
\and
\IEEEauthorblockN{Mingeun Kim}
\IEEEauthorblockA{\textit{Dept. of Information Systems} \\
\textit{Hanyang Univ.}\\
Seoul, Republic of Korea \\
alsrms0206@hanyang.ac.kr
}

}


\maketitle




\begin{abstract}
    Imagine an office worker getting ready for work in the morning, listening to music or the news through an AI speaker. During the morning routine, they might wash up in the bathroom, make coffee in the kitchen, have breakfast, choose clothes from the closet, and get dressed. For someone who moves between rooms so frequently, it's almost impossible to catch 100\% of the audio output from a stationary AI speaker.
    
    We are introducing technology that called LG Follow allows sound to follow the user, creating an environment where they can hear audio in any part of the house with LG appliances equipped with speakers.
    
    We will use a Raspberry Pi and PIR sensors to detect the user's location. The location information will be sent to a central control system, the AI speaker. For instance, if they leave the living room and enter the bedroom, the speaker in the living room will stop, and the speaker in the bedroom will automatically take over, seamlessly continuing the audio experience. Matter protocol manages the communication between Raspberry Pi and the AI speaker.
    
    Additionally, we provide an app called Sound Sketch that turns children's drawings into songs using generative AI. When a child draws a picture, the AI will create a song based on their own music. Through generative AI, the drawings will be transformed into prompts, and those prompts will be turned into music. With LG Follow, kids can enjoy listening to their own music as they move around the house, making each moment truly unique and magical.

\end{abstract}





\begin{table}[h!]
\centering
\caption{: Role Assignments}
\begin{tabular}{|c|c|p{3.5cm}|}
\hline
\textbf{Roles} & \textbf{Name} & \textbf{Task description and etc.} \\
\hline
Backend Developer & Mingyu Jung & The backend developer would implement the logic to process location data from the Raspberry Pi and PIR sensors, ensuring the user’s movement through the house is accurately tracked and relayed to the AI speaker system. Ensure that when the user moves between rooms, the appropriate speaker is activated, and the previous one is turned off, all in real-time. Handle communication between the Raspberry Pi, PIR sensors, and AI speaker system using the Matter protocol, ensuring compatibility between the smart home appliances and devices. \\
\hline

\end{tabular}
\end{table}

\begin{table}[h!]
\centering
\begin{tabular}{|c|c|p{3.5cm}|}
\hline
\textbf{Roles} & \textbf{Name} & \textbf{Task description and etc.} \\
\hline
Frontend Developer & Taegeon Park & The role of a frontend developer includes designing the frontend architecture and creating user interfaces and experiences using Figma. It involves collaborating with the backend team to discuss and implement required features while ensuring seamless integration with the server. The responsibilities also extend to reviewing designs and functionalities, keeping the user and consumer in focus to deliver an optimal user experience. \\
\hline
Backend Developer & Gyudong Kim & AI developer utilizes openai models CLIP and BLIP to generate prompts from input images. To ensure the safety of children, AI developers fine-tune the BLIP model so that inappropriate prompts are not created. Once the prompts are generated, AI developers use the Suno API to create songs from these prompts. Also AI developers are responsible for designing and building databases to store and manage various types of data, such as user information, home appliances, and the generated drawings and songs, ensuring efficient integration and management within the application. Implement a database to store user preferences, drawings, and song files generated by Sound Sketch. \\
\hline
Frontend Developer & Mingeun Kim & The frontend developer would design the interface for the ThinQ app, ensuring that both LG Follow and Sound Sketch are intuitive and user friendly. This includes creating buttons and layouts for turning speakers on/off and uploading drawings to generate songs. The developer will focus on helping LG Follow and Sound Sketch features, enabling users to control speakers in different rooms and upload or manage children's drawings in the app.  \\
\hline
\end{tabular}
\end{table}


\IEEEpeerreviewmaketitle

\section{Introduction}

\subsection{Motivation}

\subsubsection{We wanted to create an experience where the sound follows the user, ensuring uninterrupted audio no matter where they are in the house. With LG appliances equipped with speakers, the user can now enjoy continuous sound as they move from room to room, eliminating the frustration of missing important parts of the music or news.
Our motivation goes beyond just convenience, it’s about creating an immersive and seamless audio experience that adapts to the user's movement. By integrating AI and smart home appliances, we aim to make daily life smoother and more enjoyable.}
\

\subsubsection{Our motivation for making Sound Sketch from a desire to bring children's creativity to life in a magical way and to strengthen the bond between parents and their children. Children often express their imagination through drawing, and we wanted to elevate that experience by transforming their artwork into personalized music. With the power of generative AI, a simple drawing becomes a unique, original song, giving children a new way to connect with their creations.
What makes this experience even more special is the opportunity for parents and children to collaborate. By drawing together and hearing their joint artwork turned into a song, they can build memories and strengthen their connection through a shared creative process. With LG Follow, these songs can accompany them throughout the house, creating a sense of togetherness and joy, wherever they go.
Our goal is to seamlessly blend creativity and technology, offering families a fun and interactive way to bond while engaging with music that feels personal and meaningful. Every moment becomes truly unique and magical as they hear their imagination come to life.}


\
\subsubsection{We’ve made LG Follow and Sound Sketch easy to use by integrating them into the ThinQ app, ensuring both features are conveniently controlled in one place. With LG Follow, users can easily turn off the speaker in any appliance if they don’t want sound in a specific room. Meanwhile, Sound Sketch lets you transform children's drawings into songs, bringing creativity to life in a fun way. Everything is accessible and simple to control from the app, making the entire experience smooth, personalized, and user-friendly.}


\
\
\
\subsection{Problem statement (client’s needs)}

In today’s homes, AI speakers are commonly used to play music or news, but they are limited by their stationary nature. For someone who moves frequently between rooms during their morning routine—such as washing up in the bathroom, making coffee in the kitchen, or getting dressed in the bedroom—it becomes nearly impossible to catch all the audio from a single fixed speaker. This leads to a frustrating, interrupted experience where important parts of the audio are missed.

LG Follow solves this problem by allowing sound to follow the user throughout the house. As users move from room to room, LG appliances equipped with speakers provide continuous audio, eliminating gaps and ensuring they never miss a moment of music or news. This technology is designed to deliver a seamless and immersive audio experience, adapting to the user's movements to enhance their daily routine.

Additionally, we identified a need for families to engage creatively, leading to the development of Sound Sketch, an app that transforms children's drawings into personalized songs using generative AI. Children often express their imagination through drawings, and this feature elevates that creativity by turning their artwork into unique songs. It also offers parents and children the opportunity to bond through collaboration, creating memories as they hear their joint artwork come to life as music. By integrating LG Follow, these songs can follow the family throughout the house, adding a layer of joy and togetherness.

Both features, LG Follow and Sound Sketch, are integrated into the ThinQ app for ease of use, allowing users to control audio and manage drawings in one convenient place.  

\ 
\subsection{Research on any related software}

\subsubsection{Sonos (Multi-Room Audio Systems): Sonos is a leading brand in multi-room audio systems. Sonos allows users to control audio in various rooms through a smartphone app, letting users sync music in different areas of the house. However, unlike LG Follow, Sonos requires manual control for changing rooms or selecting where to play audio. It does not automatically follow the user based on their movement.}
\


\subsubsection{Google Nest and Amazon Echo: Google Nest and Amazon Echo provide smart home automation, including voice-activated music playback. They can control music in various rooms and offer smart integrations with other appliances. However, users need to manually control playback across different speakers, and the sound doesn’t seamlessly follow the user. }

\
\subsubsection{Suno: Suno is a generative AI company that specializes in music creation. The platform allows users to generate original songs, complete with melodies and harmonies, by providing simple text prompts or lyrics.}
\

\subsubsection{Melobytes: Melobytes is one tool where users can upload an image, and the system generates music from it using algorithms tailored to the visual data. It transforms the picture into unique sound compositions that reflect the visual input.}
\

\subsubsection{Img2Prompt (Anakin.ai): This tool uses AI to analyze an image and generate a text prompt that encapsulates its key visual features. The generated prompt can then be used for various creative projects, like digital art or content creation.}
\

\subsubsection{GoEnhance AI: This platform allows users to upload an image, and its AI algorithms automatically generate a text prompt based on the visual content. These prompts can be used with tools like DALL-E or Midjourney to generate new AI-created images based on the original photo's characteristics.}

\

\

\section{Requirements}

\subsection{Log in}

\subsubsection{Initial Screen: When the app is launched, the log in screen is displayed to users as the initial interface, providing access to existing users.}

\
\subsubsection{User Input: The user is required to enter their credentials, including:}
\begin{itemize}
    \item ID: A unique identifier associated with the user. \\
    \item Password: The secret passphrase known only to the user. \\
\end{itemize}

\subsubsection{Password Hashing: The entered password is securely hashed using the SHA-256 hashing algorithm before it is compared to the stored hash in the database. Hashing the password enhances security by ensuring that plain-text passwords are not transmitted or stored.}

\
\subsubsection{Validation and Authentication: The app checks if the entered ID exists in the database.}
\begin{itemize}
    \item If the ID does not exist, the app displays a message such as 'The entered ID does not exist.' \\
    \item If the ID exists, the app compares the hashed password entered by the user with the stored hash associated with the ID. \\
    \item If the hashes match, the app displays a 'Login Successful' message, indicating a successful log in. \\
    \item If the hashes do not match, the app displays an 'Incorrect password' message.\\
\end{itemize}

\subsubsection{Password Reset Option: }
\begin{itemize}
    \item For users who have forgotten their passwords, the app provides a password reset option.\\
    \item Users can initiate a password reset process by requesting an email verification. \\
    \item An email containing a verification link or code is sent to the user's registered email address. \\
    \item Upon successful verification, the user is guided through the process of resetting their password. \\
\end{itemize}


\subsection{Prompt Generation from Drawing }

\subsubsection{It generates prompts using CLIP and BLIP models from pictures entered by the user.}
\

\subsubsection{The BLIP model learns about 120,000 pieces of image captioning data in the ai-hub to generate prompts in Korean.}
\

\subsubsection{Use its own filtering function to prevent the generation of prompts that threaten children, such as cruel, dangerous, so that appropriate prompts are generated.}
\

\subsubsection{ Generated prompts are limited under 300 words.}
\begin{itemize}
    \item Get image information created by users using Amazon S3. \\
    \item Use the CLIP model to generate text that is most relevant to the image.\\
    \item Using the learned BLIP model and self-filtering functions, appropriate prompts are generated.\\
    \item Save the contents of the prompt and the date of creation to the database.\\
    \item Write the saved prompt in the description box of the song to be created.\\
\end{itemize}


\subsection{Music Generation from Prompt }

\subsubsection{It generates music using generated prompts and SUNO API.}
\

\subsubsection{Generated music contains title, description, created time, and duration with time type.}
\

\subsubsection{If user push play button, music plays and duration will be counted second by second.}
\

\subsubsection{User can download the music with download button, using Amazon S3 url.}\\

\begin{itemize}
    \item Use the prompt that saved at database.\\
    \item Transalte the Korean prompts to English with Google Translation API.\\
    \item Generate music with translated prompts and SUNO API.\\
    \item Play music with play button.\\
    \item Download the generated music, using Amazon S3.\\
\end{itemize}


\subsection{Speaker-to-Speaker Connection}

\subsubsection{Register Raspberry Pi as Matter Bridge To enable communication between Matter-compatible and non-Matter-compatible speakers, Raspberry Pi must be registered as a Matter Bridge. This allows for seamless music transitions between AI speakers and regular speakers within the Matter network. Registration procedure is as follows:}
\begin{itemize}
    \item In the Matter-supported device management app, select the option to add a new Matter Bridge.\\
    \item Scan the QR code or manually enter the setup code provided by the Raspberry Pi.\\
    \item Once connected, Raspberry Pi will act as a Matter Bridge between devices.\\
    \item Raspberry Pi is now registered as the Matter Bridge.\\
\end{itemize}


\subsubsection{Control Speakers Through Raspberry Pi Once the Raspberry Pi is set as a Matter Bridge, it facilitates communication between regular speakers and AI speakers in the home network. Matter protocol handles real-time status updates and control commands.\\\\Procedure:}
\begin{itemize}
    \item When switching music playback from AI speakers to regular speakers, Raspberry Pi controls this transition using Matter protocol.\\
    \item Music playback transitions in real-time, and Raspberry Pi ensures that both AI speakers and non-Matter speakers can communicate smoothly.\\
\end{itemize}


\subsection{User Location Tracking}

\subsubsection{Register PIR Sensor to Raspberry Pi PIR sensors detect user movement and must be connected to Raspberry Pi for real-time location tracking and music control.
Registration procedure is as follows:}
\begin{itemize}
    \item Connect the PIR sensor to Raspberry Pi.\\
    \item Add the PIR sensor to the Raspberry Pi through  directly register the sensor.\\
    \item PIR sensor is now registered to Raspberry Pi for location tracking.\\
\end{itemize}


\subsubsection{Control Based on Location Tracking The PIR sensor tracks the user’s location, and the Raspberry Pi uses this data to send streaming transition commands to the AI speakers or room speakers based on movement.}
\begin{itemize}
    \item When the user moves to another room, the Raspberry Pi sends the command to switch music playback to the speakers in the new room.\\
    \item Matter protocol ensures the transition occurs in real-time based on location tracking data provided by the PIR sensor.\\
\end{itemize}


\subsection{Location-Based Sound Transition}

\subsubsection{Register Room Speakers for RTP Streaming Room speakers need to be set up for RTP (Real-time Transport Protocol) audio streaming to ensure real-time music transitions. \\\\Procedure:}
\begin{itemize}
    \item Set up room speakers for RTP streaming.\\
    \item Ensure that the speakers are registered to receive RTP streams from the Raspberry Pi.\\
\end{itemize}

\subsubsection{Handle Music Transitions via RTP and Matter Music transitions are handled using RTP for audio data transfer, while Matter protocol manages communication between speakers.}
\begin{itemize}
    \item When music moves from the living room to another room, RTP streams the audio to the room speaker.\\
    \item Matter ensures that the correct speaker is selected based on the user’s movement.\\
\end{itemize}


\subsection{Speaker On/Off Functionality}

\subsubsection{Register Central Server for User Preferences To manage user preferences for music playback in certain rooms, the central server must communicate with the Raspberry Pi via MQTT and Matter.}
\begin{itemize}
    \item Configure the central server to store user preferences for music playback in each room.\\
    \item Use MQTT to send this configuration to the Raspberry Pi.\\
\end{itemize}

\subsubsection{Transmit User Preferences via MQTT serves as the lightweight message protocol to send the user's settings from the central server to the Raspberry Pi.}
\begin{itemize}
    \item The central server sends an MQTT message containing the user's preferences (e.g., turn off music in a specific room).\\
    \item Raspberry Pi receives the MQTT message and applies the preferences.\\
\end{itemize}

\subsubsection{Control Music Playback via Matter Raspberry Pi uses Matter to control music playback based on the user’s settings.}
\begin{itemize}
    \item If the user prefers no music in a specific room, Raspberry Pi prevents playback in that room.\\
    \item Matter ensures the speakers are controlled accordingly.\\
\end{itemize}


\subsection{Image Temporary Storage and Music CRUD}

\subsubsection{Image Temporary Storage}
\begin{itemize}
    \item When a user saves a drawing temporarily, Redis, a fast-access cache database, is used to store the image data. \\
    \item Redis acts as a memory-based storage solution that saves the temporary image data and allows for rapid retrieval when needed. This is particularly useful if the user pauses or edits the drawing, enabling them to resume their work later without data loss. \\
    \item By utilizing Redis, the app ensures quick access to the saved image while reducing latency and enhancing user experience during the drawing process.\\
\end{itemize}


\subsubsection{Music CRUD}
\begin{itemize}
    \item Once the drawing is transmitted to the Flask server, the image is stored temporarily in Redis, a memory-based cache, which allows fast retrieval of the drawing. This enables the user to pause or resume work on the drawing.\\
    \item The Flask server processes the request and ensures real-time streaming of the music, allowing users to listen to the song as soon as it's generated.\\
    \item For storage and long-term management of the generated music, the metadata and music links are handled by Spring Boot, which communicates with a MySQL database to store information such as the song title, date of creation, and the associated drawing theme.\\
    \item Users can perform CRUD operations on the generated music.
\begin{itemize}
    \item Create: Saving the generated music to the database and S3 for long-term access.\\
    \item Read: Streaming the saved music from S3 and retrieving metadata from the database for display in the app.\\
    \item Update: Modifying the song’s metadata or regenerating the music by submitting a new drawing.\\
    \item Delete: Removing the music file from S3 and its metadata from the MySQL database, while ensuring Redis invalidates any cached data related to the deleted music.\\
\end{itemize}
\end{itemize}

\

\section{Development environment}

\subsection{Choice of software development platform}


\end{document}


